%
% SecIntro: Introduction section
% This section mainly summarizes the project proposal/problem statement
%

For our project, we propose Mobile Vital Radio (MoViRad), a novel method of measuring physiological signals using a mobile device. We are especially interested in measuring the heartbeat and breathing rate of a person using frequency modulated continuous waves (FMCW). While past work exists that detects the breathing rate through use of a mobile device~\cite{nandakumar_mobisys15}, they were ultimately unable to get down to the fine resolution required to measure heart beats. On the other hand, the heartbeat and breathing rate detection is achieved in Vital-Radio~\cite{Adib_acm15} by extracting the heart rate based on ballistocardiography (BCG). BCG refers to body movement that is synchronous with heart beat due to ventricular pump activity; in this case, the movement of interest is the breathing, or chest movement. As a result, we expect to obtain heart beat measurement by observing small fluctuation on top of breathing. Such minute yet periodic movement could be extracted by applying Fourier transform (FFT) of the breathing signal, and identify corresponding peaks in the frequency spectrum as breathing is typically measured of sub-hertz frequency while heart beat should be in the vicinity of tens of hertz. However, Vital-Radio could not be directly applied on top of~\cite{nandakumar_mobisys15} since the medium is acoustic wave in contrast to RF signal. As a result, the sampling rate required (in other words, the frequency bin resolution of FFT) is not available. In view of the limitation presented by acoustic wave, we propose Mobile Vital Radio (MoViRad) to precisely measure the phase of received FMCW by performing interpolation on individual bins in the obtained FFT. 

The rest of the report is organized as follows. Section~\ref{sec:movirad} illustrates the development of our proposed method by presents a quick overview of FMCW followed by analysis of the FFT interpolation method. Section~\ref{sec:meas} presents the experimental data we have obtained so far. Finally, Section~\ref{sec:midreport} summarizes our progress and shows the to-do-lists to accomplish the project.